\documentclass[preprint, 12pt]{elsarticle}

% KU requirements
\usepackage[margin=2.5cm]{geometry}
\usepackage{setspace}
\onehalfspacing

% Temporary packages used while editing
\usepackage[usenames,dvipsnames]{color}

% Packages
\usepackage[colorlinks=true]{hyperref}
\usepackage{xspace}

% Options
\biboptions{authoryear}

% Custom commands
\newcommand{\Cree}{\emph{Cree}\xspace}

\begin{document}

\begin{frontmatter}

\title{\emph{Cree}: A modern toolbox for readymade economic experiments}
\author{Jonas K. Sekamane}
\journal{Supervised by Ulrik Haagen Nielsen.}
\address{{\color{red} First draft}}

\begin{abstract}
{\color{red} ...}
\end{abstract}
%\begin{keyword}Science \sep Publication \sep Complicated\end{keyword}

\end{frontmatter}


%% main text
\section{Introduction}
\label{S:Introduction}

This paper introduces a modern toolbox for readymade economic experiments called \Cree. \Cree takes advantage of the great advances in technology, in particular the new types of devices (smart-phones, tablets) and the availability of general-purpose software libraries. 

The great merit of \Cree is the very few restrictions it places on the equipment facing subjects. Few restrictions clear the way for much broader participation, strengthening the external validity of any experiment. \Cree experiments can be run in a myriad of settings, including laboratories, classrooms, workplaces, or over the Internet. With \Cree the computer-equipped laboratory is no longer a necessity -- lowering the overall costs of conducting economic experiments. The researcher can still choose to provide subjects with devices, but can just as easily let subjects use their own devices. 

\Cree is build using web technologies. Content is structured using the markup language \emph{HTML}. The layout and presentation across different screen sizes is archived with the style sheet language \emph{CSS}. And all logic is constructed using the programming language \emph{JavaScript}. All web browsers interpret these three cornerstone languages and render pages accordingly. Subjects participate in \Cree experiments though a web browser. Because \Cree is fundamentally native to the web browser, it avoids many of the restrictions, that other toolboxes suffer from.

In addition there is a vibrant ecosystem surrounding these web technologies. \Cree exploits this ecosystem and takes full advantage of the software libraries that exist. This provides stability, flexibility, and makes it easy to set up and run economic experiments. Fore instances, \Cree uses the software library \emph{Node.js} to run the server and handle networking issues. The asynchronous architecture of \emph{Node.js} gives \Cree the flexibility to handle real-time events, which many other toolboxes is not capable of. With this flexibility \Cree can run everything, from simple dictator game experiments, to highly sophisticated auction experiments. {\color{Bittersweet}[Another example is the software library \emph{Bootstrap} developed by Twitter, which is used to scale pages so they match the respective screen sizes. Thereby bypassing the need to design separate versions for respectively smart-phones, tablets and computers. In sum,]} \Cree is a framework that provides the tools to handle many of the otherwise mundane, complicated or manually tasks required to set up and run economic experiments.

Technological advances opens up a new path, however the path is not without obstacles. This paper explores and discusses how to handle these obstacles. Some obstacles are alleviated through appropriate technical design of the toolbox. Other obstacles require actions taken by the researcher.

\section{Requirements}
\label{S:Requirements}

\subsection{What are the alternative toolboxes and how do they compare to cTree?}

*z-Tree* is currently the commonly used toolbox for conducting economic experiments. The development of *z-Tree* started in 1995 and has been updated continuously \cite{Fischbacher_2007}. However the foundation and guiding principles of *z-Tree* is based on the technology that was available two decades ago.

\section{Running experiments}
\label{S:Running}

\section{Expected results}
\label{S:Results}

\section{Designing experiments}
\label{S:Designing}

\section{Conclusion}
\label{S:Conclusion}


%% References
\bibliographystyle{apalike}
\raggedright
\singlespacing
\bibliography{references.bib}

\end{document}